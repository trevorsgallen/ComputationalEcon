\documentclass{beamer}
\usepackage{graphicx}
\usepackage{tikz}
\usetikzlibrary{shapes,arrows}
\usepackage{tikz}
\usetheme{default}
%\usecolortheme{seahorse}
\usepackage[normalem]{ulem}

  \setbeamertemplate{footline}[page number]
\setbeamertemplate{navigation symbols}{}
\setbeamertemplate{frametitle}[default][center]
\setbeamerfont{frametitle}{shape=\scshape}

\usepackage{color}

{\title{\textsc{Numerical Methods-Lecture I: Outline of Course}}
\author{Trevor Gallen}
\date{}

\begin{document}


\begin{frame}
\titlepage
\end{frame}

\begin{frame}
\frametitle[alignment=center]{Goals}
Aim is to teach numerical methods, give you the tools you need to write down, solve, and estimate models
\bigskip
\begin{enumerate}
\item Interpolation 
\bigskip
\item Numerical derivatives
\bigskip
\item Maximization/minimization 
\bigskip
\begin{itemize}
\item Deterministic, stochastic 
\bigskip
\item Derivative-based, derivative-free
\bigskip
\item Local, global
\bigskip
\end{itemize}
\item \sout{Numerical integration/quadrature}
\bigskip
\item Bellman equations 
\bigskip
\item  Reinforcement Learning
\bigskip
\item Calibrate (and possibly estimate) structural models 
\end{enumerate}
\end{frame}

\begin{frame}
\frametitle[alignment=center]{Odds \& Ends}
\begin{enumerate}
\item This course runs for 8 weeks, from October 20th-December 8th(ish).\\
\bigskip
\item Office hours are Thursday mornings, 8:50-9:50 a.m. in KRAN 315
\bigskip
\item Contact:  tgallen [at] purdue
\bigskip
\item Grading:  Four homeworks, one ``paper"/model
\bigskip
\item Course Text:  Judd
\bigskip
\item Also useful:  Miranda \& Fackler
\bigskip
\item Various readings
\end{enumerate}
\end{frame}

\begin{frame}
\frametitle[alignment=center]{Background on Computational}
\begin{itemize}
\item More and more, interesting problems have wrinkles
\bigskip
\item Simple examples:
\bigskip
\begin{itemize}
\item Game theory (Bringing game parameters to data)
\bigskip
\item Industrial organization (Demand system estimation)
\bigskip
\item Labor economics (Household bargaining, nonlinear constraints)
\smallskip
\item Public economics (program participation, dynamics)
\bigskip
\item Macroeconomics (DSGE models of last 30 years)
\end{itemize}
\end{itemize}
\end{frame}

\begin{frame}
\frametitle[alignment=center]{Distinguishing characteristics}
\begin{itemize}
\item Explicit specifications of preferences, production, and behavior
\smallskip
\item Frequently, many different actors 
\bigskip
\item Frequently, markets clearing
\bigskip
\item Numerical output
\bigskip
\item Increasingly, dynamic
\end{itemize}
\end{frame}


\begin{frame}
\frametitle[alignment=center]{Great Leap Forward}
\begin{itemize}
\item Focus on numerical output has been great!
\bigskip
\begin{itemize}
\item Complexity \only<2->{\color{red}{$\neq$ black box!!!}\color{black}}
\bigskip
\item No more hand waving (or less)
\bigskip
\item Closer link to data
\bigskip
\item Failure of models is feature not bug\only<3->{\color{red}{(?!)}\color{black}}
\bigskip
\item Real predictions
\bigskip
\end{itemize}
\item But it has its costs
\bigskip
\begin{itemize}
\item Complexity 
\bigskip
\item Death of economic intuition
\bigskip
\item Closed form
\bigskip
\item Unclear if many numerical heuristics work
\bigskip
\item Perhaps most importantly: black hole of time!
\end{itemize}
\end{itemize}
\end{frame}

\begin{frame}
\frametitle[alignment=center]{Outline of course}
\begin{itemize}
\item Matlab introduction
\bigskip
\item Bellman equations: theory
\bigskip
\item Bellman equations: extremely limited numerical solution
\bigskip
\item Numerical derivatives
\bigskip
\begin{itemize}
\item Derivative-based and derivative-free
\bigskip
\item Local and global
\bigskip
\end{itemize}
\item Maximization
\bigskip
\item Equation solving
\bigskip
\item Interpolation
\bigskip
\item Integration
\bigskip
\item Simulated methods of estimation
\end{itemize}
\end{frame}

\begin{frame}
\frametitle[alignment=center]{Potential Uses of concepts}
\begin{itemize}
\item Bellman equations: most dynamic problems
\bigskip
\item Numerical derivatives: maximization, equation-solving
\bigskip
\item Maximization: Agent problems, estimation
\bigskip
\item Equation solving: Solving models
\bigskip
\item Interpolation: Making your life easier, allowing for richer agent choice, better estimation
\bigskip
\item Integration: Allowing for shocks, allowing for agent heterogeneity
\end{itemize}
\end{frame}


\begin{frame}
\frametitle[alignment=center]{What do you want to see?}
 What tools, models, papers, methods would you like to learn?
\end{frame}






\end{document}
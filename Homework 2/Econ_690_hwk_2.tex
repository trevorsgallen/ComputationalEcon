\documentclass[11pt]{article}
\usepackage{appendix}
\usepackage{graphicx} 
\usepackage{setspace}
\usepackage{amsmath,float,verbatim,multicol}
\usepackage{array}
\usepackage{lscape}
\usepackage{hyperref}

\usepackage{amssymb}
\bibliographystyle{plainnat}
\usepackage[round]{natbib}
\usepackage{multirow}

\setlength{\textheight}{8.8in} \setlength{\textwidth}{6.3in}
\setlength{\oddsidemargin}{0.2in} \setlength{\topmargin}{-0.30in}
\setlength{\footnotesep}{10.0pt}

\newcommand{\ol}{\overline}



\renewcommand{\baselinestretch}{1.25}
\title{Solving a Stochastic Lifecycle Model}
\author{ Trevor Gallen \\ Econ 64200 }
\date{Fall 2022}

\begin{document}
\bibliographystyle{myplainnat}
%\bibpunct{(}{)}{;}{a}{}{,}6868

\maketitle

This homework is the first part of a multi-part homework.  This homework starts us with a relatively simple deterministic lifecycle model and asks you to solve it in Matlab.\\

\textbf{Deliverables}
\begin{itemize}
\item You should have a word/\LaTeX document that has three sections: 
\begin{enumerate}
\item Discusses the model and answers the questions I pose throughout.
\item Contains the tables and figures you will produce.
\item Contains a discussion of your programming choices if you had to make any.
\end{enumerate}
\item You should have a Matlab file or set of files (zipped) that contain \textbf{all} your programs and raw data.  There should be a file called ``Main.M" that produces everything I need in one click.
\end{itemize}
\ \\
\textbf{Note}: I've deliberately given you a  hard problem, and it might be classified as ``too hard."   Try solving simple versions of the problem first.  A lot of computational is thinking about how to simplify problems while still getting at the core question, and please exercise your ability to do so here!  \textbf{Start small}.  Also note that there are a few theoretical model loose ends I don't mention.  If you notice them, let me know how you solved them in the solutions!\\
\ \\


\section{Blundell, Preston, and Pistaferri (2008)}
Read through Blundell, Pistaferri, and Preston (2008), paying particular attention to Section II A, B, and C.  They write out a reduced-form model of an income and consumption process, and estimate partial insurance from it.  In this homework, we'll take a first attempt at solving a simple VFI model in the same vein.  In doing so, we will be committing a similar exercise as Kaplan and Violante (2010).  We will return to this in another homework, so it's worth taking a look at the paper.


\section{Model}
Households have Stone-Geary utility over consumption $c_t$:
$$u(c_t,L_t)=\sqrt(c_t-1)$$
Their income each period is the sum of permanent income $P_t$ and a transitory shock $\epsilon_t$:
$$ Y_t=P_t+\epsilon_t$$
Where:
$$\epsilon_t\sim\mathcal{N}(0,\sigma^2_\epsilon)$$
Permanent income is a random walk:
$$P_t=P_{t-1}+\zeta_t$$
Where:
$$\zeta_t\sim\mathcal{N}(0,\sigma^2_\zeta)$$
Households face the budget constraint:
$$Y_t+(1+r)s_{t-1}=c_t+s_t$$
And the borrowing constraint:  $s_t\geq \overline{s}$.\\

Households maximize the net present value of utility, discounted at a rate $0<\beta<1$. Assume that households live until period $T$.\\

Now, let the following numerical assumptions hold:
\begin{table}[ht!]
\centering
\begin{tabular}{lcc}
\hline
\hline
\multicolumn{3}{c}{Table 1: Calibration}\\
\hline
Concept & Parameter & Value \\ 
Lifespan & T & 45 \\
Discount factor & $\beta$ & 0.96\\
Borrowing constraint & $\overline{s}$ & -0.2 \\
Initial permanent income:  & $P_0$ & 5\\
Variance of permanent income shock:  & $\sigma^2_\nu$ & 0.02\\
Variance of transitory income shock:  & $\sigma^2_\zeta$ & 0.04\\
Interest rates:  & $r_{t}$ & 0.05\\
\hline
\hline
\end{tabular}
\end{table}

\textbf{Question 1:} Write down and solve the value function problem the household faces.  There are many ways to write the problem, but one is to have four state variables:  $T$, $P$, $\epsilon$, and $s$.  Note that some states ($P$, $\epsilon$) are unbounded.  Feel free to bound them in such a way that your results will not change in any important way (think about how to do this!)\\
\ \\
\textbf{Question 2:} Simulate 100 households, and use that data to calculate the variance of permanent and transitory shocks, as well as the insurance coefficients given by Equations (C1)-(C4) in Blundell, Preston, and Pistaferri (2008).  You do not have to do formal GMM.  How do your results match up our calibration?  To their estimates?  \\



\clearpage
\section{Research}
You now have a relatively simple numerical model of people's savings in response to transitory and permanent shocks.  However, the model is too simple to be particularly interesting.  Change the model in an interesting way, and report your results.  This is your opportunity to actually think, so please take time with this:  how can you adjust the model, parameterization, etc. to make this model (a) more interesting and (b) more realistic?

\end{document}






\documentclass{beamer}
\usepackage{graphicx}
\usepackage{tikz}
\usetikzlibrary{shapes,arrows}
\usepackage{tikz}
\usetheme{default}
%\usecolortheme{seahorse}
\usepackage{default}

  \setbeamertemplate{footline}[page number]
\setbeamertemplate{navigation symbols}{}
\setbeamertemplate{frametitle}[default][center]
\setbeamerfont{frametitle}{shape=\scshape}

\usepackage{color}

{\title{\textsc{Numerical Methods-Lecture IV: Bellman Equations: Solutions}}
\author{Trevor Gallen}
\date{}

\begin{document}


\begin{frame}
\titlepage
\end{frame}

\begin{frame}
\frametitle[alignment=center]{Preliminaries}
\begin{itemize}
\item We've seen the abstract concept of Bellman Equations
\bigskip
\item Now we'll talk about a way to solve the Bellman Equation: Value Function Iteration
\bigskip
\item This is as simple as it gets!
\end{itemize}
\end{frame}

\begin{frame}
\frametitle[alignment=center]{Value Function Iteration}
\begin{itemize}
\item Bellman equation:
$$V(x)=\underset{y\in\Gamma(x)}{\max}\left\{F(x,y)+\beta V(y)\right\}$$
\item A solution to this equation is a function $V$ for which this equation holds $\forall$ x
\item What we'll do instead is to assume an initial $V_0$ and define $V_1$ as:
$$V_1(x)=\underset{y\in\Gamma(x)}{\max}\left\{F(x,y)+\beta V_0(y)\right\}$$
\item Then redefine $V_0=V_1$ and repeat
\item Eventually, $V_1\approx V_0$
\begin{itemize}
\item But $V$ is typically continuous: we'll discretize it
\item Make function continuous by connecting the dots
\end{itemize}
\end{itemize}
\end{frame}

\foreach \x in {1,2,...,15}{
\begin{frame}
\frametitle[alignment=center]{Aside: Approximating f(x) }
\begin{figure}
\centering
\includegraphics[scale=0.5]{interpfig_\x.png}
\end{figure}
\end{frame}
}

\begin{frame}
\frametitle[alignment=center]{Basic Steps}
\begin{enumerate}
\item Choose grid of states $X$ and a stopping threshold $\epsilon$
\item Assume an initial $V_0$ for each $x\in X$
\item For each $x\in X$, solve the problem:
$$\underset{y\in\Gamma(x)}{\max}\left\{F(x,y)+\beta V_0(y)\right\}$$
\item Store the solution as $V_1(x)$
\item Redefine $V_0=V_1$
\item Repeat steps 3-5 until $abs(V_1-V_0)<\epsilon$. 
\item Now, for all your relevant points, the Bellman equation holds
\item Solve the system one last time, storing the policy function
\end{enumerate}
\end{frame}

\begin{frame}
\frametitle[alignment=center]{How do I solve the problem?}
\begin{itemize}
\item Step 3 requires you to solve:
$$\underset{y\in\Gamma(x)}{\max}\left\{F(x,y)+\beta V_0(y)\right\}$$
\item How do we do it?  
\item How do we maximize?
\item We'll learn good ways
\item For now, discretize all your choices like you discretized your states
\item Pick best choice, store utility
\item If you allow for choices to imply states that aren't defined, interpolate linearly
\end{itemize}
\end{frame}

\begin{frame}
\frametitle[alignment=center]{Aside: Intuition for VFI}
\begin{itemize}
\item In the iteration period, all future states are the same: we don't care what happens.
\item In a ``cake-eating" example, this means eat everything.
\item In such a scenario, we eat all the cake: we're happier with more cake.
\item When we iterate once more, now tomorrow is the last day on earth: we now prefer saving a little cake.
\item When we iterate again, tomorrow's tomorrow is the last day...
\item Because we discount, as we iterate more, whatever we do on the last day matters less and less
\item Eventually, we're all but immortal: $\underset{t\rightarrow\infty}\lim\beta^t=0$ \uncover<2->{(really, $\underset{t\rightarrow\infty}\lim\beta^tu_2(x_t,x_{t+1})x_{t+1}=0)$}
\end{itemize}
\end{frame}


\begin{frame}
\frametitle[alignment=center]{Let's do a concrete example}
$$U(c_t)=\log(c_t)$$
$$c_t+i_t=k_t^{0.7}$$
$$k_{t+1}=0.93 k_t+i_t$$
\begin{itemize}
\item Discretize states
\begin{itemize}
\item Minimum: $\underbar{k}=0$
\item Maximum: $\bar{k}=0.93 \bar{k}+\bar{k}^{0.7}\Rightarrow \bar{k}=7075$
\item Choose 10 possible steps
\end{itemize}
\item Allow choice of feasible discrete $k$
\item Choose best, store it.
\item Repeat
\end{itemize}
\end{frame}

\begin{frame}
\frametitle[alignment=center]{Solving in Matlab}
\scriptsize
\texttt{alpha = 0.7;}\\
\texttt{delta = 0.07;}\\
\texttt{k\_min = 0;}\\
\texttt{k\_max = 7075;}\\
\texttt{k\_num = 10;}\\
\texttt{k\_space = linspace(k\_min,k\_max,k\_num);}\\
\texttt{V\_1 = 0.*k\_space;}\\
\texttt{V\_0 = V\_1;}\\
\texttt{error = Inf;}\\
\texttt{while error > 1e-10}\\
\ \ \ \texttt{for k\_index = 1:k\_num}\\
\ \ \ \ \texttt{k = k\_space(k\_index);}\\
\ \ \ \ \texttt{kchoice\_index = find(k\_space $<$ 0.93k+k.\string^0.7);}\\
\ \ \ \ \texttt{k\_choices = k\_space(kchoice\_index);}\\
\ \ \ \ \texttt{c\_choices = 0.93*k+k.\string^0.7-k\_choices;}\\
\ \ \ \ \texttt{utility = log(c\_choices) + beta V\_0(find(kchoice\_index));}\\
\ \ \ \ \texttt{[V,ind] = max(utility);}\\
\ \ \ \ \texttt{V\_1(k\_index) = V;}\\
\ \ \ \ \texttt{k\_best(k\_index) = k\_choices(ind);}\\
\ \ \ \ \texttt{end}\\
\ \ \texttt{error = max(abs(V\_1-V\_0))}\\
\texttt{end}
\end{frame}


%\begin{frame}
%\frametitle[alignment=center]{ What are we doing?}
%\begin{itemize}
%\begin{figure}
%\centering
%\includegraphics{VFI_Simple_1.png}
%\end{figure}
%\end{itemize}
%\end{frame}

%\begin{frame}
%\frametitle[alignment=center]{ What are we doing?}
%\begin{itemize}
%\begin{figure}
%\centering
%\includegraphics{VFI_Simple_2.png}
%\end{figure}
%\end{itemize}
%\end{frame}

%\begin{frame}
%\frametitle[alignment=center]{ What are we doing?}
%\begin{itemize}
%\begin{figure}
%\centering
%\includegraphics{VFI_Simple_3.png}
%\end{figure}
%\end{itemize}
%\end{frame}

%\begin{frame}
%\frametitle[alignment=center]{ What are we doing?}
%\begin{itemize}
%\begin{figure}
%\centering
%\includegraphics{VFI_Simple_4.png}
%\end{figure}
%\end{itemize}
%\end{frame}

%\begin{frame}
%\frametitle[alignment=center]{ What are we doing?}
%\begin{itemize}
%\begin{figure}
%\centering
%\includegraphics{VFI_Simple_5.png}
%\end{figure}
%\end{itemize}
%\end{frame}


\begin{frame}
\frametitle[alignment=center]{Simulating in Matlab}
\scriptsize
\texttt{num\_i = k\_num}\\
\texttt{num\_t = 50;}\\
\texttt{k\_sim = NaN(num\_i,num\_t);}\\
\texttt{k\_sim(:,1) = NaN(num\_i,num\_t);}\\
\texttt{for i = 1:num\_i}\\
\texttt{for t = 1:num\_t}\\
\texttt{k\_sim(i,t+1) = k\_best(find(k\_space)==k\_sim(i,t))}\\
\texttt{end}
\texttt{end}
\end{frame}

%\begin{frame}
%\frametitle[alignment=center]{ What are we doing?}
%\begin{itemize}
%\begin{figure}
%\centering
%\includegraphics{VFI_Simple_Sim_1.png}
%\end{figure}
%\end{itemize}
%\end{frame}

%\begin{frame}
%\frametitle[alignment=center]{ What are we doing?}
%\begin{itemize}
%\begin{figure}
%\centering
%\includegraphics{VFI_Simple_Sim_2.png}
%\end{figure}
%\end{itemize}
%\end{frame}


\end{document}
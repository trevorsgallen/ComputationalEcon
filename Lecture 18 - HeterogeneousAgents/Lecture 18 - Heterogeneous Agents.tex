\documentclass{beamer}
\usepackage{graphicx}
\usepackage{tikz}
\usetikzlibrary{shapes,arrows}
\usepackage{tikz}
\usetheme{default}
%\usecolortheme{seahorse}
\usepackage{default}
  \setbeamertemplate{footline}[page number]
\usepackage{multirow}
\setbeamertemplate{navigation symbols}{}
\setbeamertemplate{frametitle}[default][center]
\setbeamerfont{frametitle}{shape=\scshape}

\usepackage{xcolor}

\usepackage[flushleft]{threeparttable}

{\title{\textsc{Heterogeneous Agents} \\ \tiny (See Krusell Smith 1998)}
\author{Trevor Gallen}
\date{}
\begin{document}
\renewcommand*{\inserttotalframenumber}{\pageref{lastframe}}

\begin{frame}
\titlepage
\end{frame}

\begin{frame}
\frametitle[alignment=center]{Introduction}
\begin{itemize}
\item We spend an enormous time on representative agents
\bigskip
\item Model has been quite fruitful
\bigskip
\item But there are theoretical reasons to think that a RA model wouldn't capture everything
\bigskip
\item What about heterogeneity?  Income constraints?
\end{itemize}
\end{frame}

\begin{frame}
\frametitle[alignment=center]{Krusell Smith}
\begin{itemize}
\item Take same basic NCG model we've been using
\bigskip
\item We don't care who owns what: only the \emph{total} income and capital in the society matter
\bigskip
\item It's plausible to think that the \emph{distribution} matters
\bigskip
\item Now, people not only face aggregate uncertainty but also idiosyncratic income\/employment shocks, and that they can't borrow past an exogenously-set lower bound.
\bigskip
\item Because you can't insure your shocks, there's a wealth distribution
\end{itemize}
\end{frame}

\begin{frame}
\frametitle[alignment=center]{The Environment}
\begin{itemize}
\item People have preferences over their stream of consumption $c_t$:
$$E_0\sum_{t=0}^\infty \beta^tU(c_t)$$
\item With:
$$U(c)=\underset{\nu\rightarrow \sigma}{\lim}\frac{c^{1-\nu}-1}{1-\nu}$$
\item Aggregate production $y$:
$$y=c+k'-(1-\delta)k$$
\item Labor supplied is $\epsilon\tilde{l}$, where $\epsilon\in\{0,1\}$ is exogenous
\bigskip
\item Also have aggregate shock $z\in\{b,g\}$, correlated with $\epsilon$
\end{itemize}
\end{frame}

\begin{frame}
\frametitle[alignment=center]{The Environment: Shocks}
\begin{itemize}
\item Probability transition $\pi_{ss'\epsilon\epsilon'}$, denotes your probability of moving to state $s'$ from state $s$ and at the same time to state $\epsilon'$ from state $\epsilon$.  
\bigskip
\item All inflows/outflows are balanced, so that (conditioning on $z$), we have independence across individuals
$$\pi_{ss'00+\pi_{ss'01}}=\pi_{ss'10}+\pi_{ss'11}=\pi_{ss'}$$
\item<2-> That is, $\epsilon$ today doesn't impact $s$ transition probabilities
\bigskip
\item<3-> In addition, the aggregate number of households in the bad state is always $u_g$ or $u_b$, depending on the state:
$$u_s\frac{\pi_{ss'00}}{\pi_{ss'}}+(1-u_s)\frac{\pi_{ss'10}}{\pi_{ss'}}=u_{s'}$$
\end{itemize}
\end{frame}

\begin{frame}
\frametitle[alignment=center]{State Variables-I}
\begin{itemize}
\item There is only one asset: aggregate capital
\bigskip
\item Denoting aggregate capital as $\bar{k}$ and aggregate labor as $\bar{l}$:
$$w(\bar{k},\bar{l},z)=(1-\alpha)z\left(\frac{\bar{k}}{\bar{l}}\right)^\alpha\ \ \ \ \ r(\bar{k},\bar{l},z)=\alpha z\left(\frac{\bar{k}}{\bar{l}}\right)^{\alpha-1}$$
\item In order to know what $w$ and $r$ will be, I need to know... 
\bigskip
\item ...what $\bar{k}$ and $\bar{l}$ will be!
\bigskip
\item $\bar{k}$ and $\bar{l}$ come from everyone...I need to know the distribution of capital by employment status, called $\Gamma$, as well as my standard $z$.
\end{itemize}
\end{frame}

\begin{frame}
\frametitle[alignment=center]{State Variables-II}
\begin{itemize}
\item I need to know the distribution of capital, $\Gamma$
\bigskip
\item To plan for tomorrow, I need to know the \emph{law of motion} of the distribution, to find $\Gamma'$.  
\bigskip
\item Call this law of motion of the distribution $H(\gamma,z,z')$
\bigskip
\item Then for an individual, he needs to know his own capital, his own employment, the distribution of capital, and aggregate productivity: $(k,\epsilon,\Gamma,z)$
\end{itemize}
\end{frame}

\begin{frame}
\frametitle[alignment=center]{Optimization Problem}
\begin{itemize}
\item The agent's optimization problem is therefore:
$$V(k,\epsilon,\Gamma,z)=\underset{c,k'}{\max}\left[U(c)+\beta E(V(k',\epsilon';\Gamma,z'))\right]$$
\item Subject to:
$$c+k'=r(\bar{k},\bar{l},z)k+w(\bar{k},\bar{l},z)\tilde{l}\epsilon+(1-\delta)k$$
$$\Gamma'=H(\Gamma,z,z')$$
$$k'\geq 0$$
\item Solving this problem, we get:
$$k'=f(k,\epsilon,\Gamma,z)$$
\end{itemize}
\end{frame}

\begin{frame}
\frametitle[alignment=center]{Equilbrium}
Equilibrium is:
\begin{enumerate}
\item $H$, the law of motion for $\Gamma$, consistent with $f$
\bigskip
\item $V$ and $f$, the individual's value and policy functions
\bigskip
\item $r$ and $w$, pricing functions that clear markets given the consumer's $V$ and $f$
\end{enumerate}
Do you see the problem?
\end{frame}

\begin{frame}
\frametitle[alignment=center]{A solution(?)}
\begin{itemize}
\item How can we characterize a distribution?
\bigskip
\item Only give the agents the first $m$ (statistical!) moments of the distribution and make their best guess
\bigskip
\item But then...we still don't have a good law of motion, consistent with $f$?
\end{itemize}
\end{frame}

\begin{frame}
\frametitle[alignment=center]{A solution algorithm}
\begin{enumerate}
\item Summarize distribution by first $m$ statistical moments
\bigskip
\item \emph{Assume} a law of motion for agents
\bigskip
\item Solve and simulate behavior (inner loop)
\bigskip
\item From simulated behavior, solve for new law of motion.  
\bigskip
\item If new law of motion is different, go back to step (3).  Otherwise, proceed. 
\bigskip
\item If result is different from with $m-1$ moments, add a moment.  If not, end.
\end{enumerate}
\end{frame}

\begin{frame}
\frametitle[alignment=center]{Model parameters}
Assume some parameters
\begin{itemize}
\item Period of one quarter
\item $\beta=0.99$
\item CRRA $\sigma=1$
\item Capital share $\alpha=0.36$
\item Good and bad shock: $z_g=1.01$ \& $z_b=0.99$
\item Unemployment rates: $u_g=0.04$ \& $u_b=0.10$
\item Choose process for $(z,\epsilon)$ so:
\begin{itemize}
\item Average duration of good and bad times is 8 quarters
\item Average duration of an unemployment spell is 1.5 quarters in good times and 2.5 quarters in bad times
\end{itemize}
\end{itemize}
\end{frame}


\begin{frame}
\frametitle[alignment=center]{Results: Approximate Aggregation}
Assume some parameters
\begin{itemize}
\item Only the mean of capital matters, predicts 99.9998\% of variation in capital 
\bigskip
\item Better prediction techniques would mean nothing
\bigskip
\item Caution: self-fulfilling approximate equilbiria \emph{might} exist...
\bigskip
\item But no evidence for this
\end{itemize}
\end{frame}

\begin{frame}
\frametitle[alignment=center]{Results: Why only the mean?}
Assume some parameters
\begin{itemize}
\item Fundamentally, all that matters is your propensity to save out of wealth 
\bigskip
\item If everyone always saves the same proportion of wealth, it doesn't matter who has the wealth
\bigskip
\item Savings behavior is only atypical for the very poor
\bigskip
\item But the really poor don't matter for aggregate capital
\end{itemize}
\end{frame}

\begin{frame}
\frametitle[alignment=center]{Some issues}
Assume some parameters
\begin{itemize}
\item Model distribution (entirely endogenous from labor) is not skewed enough 
\bigskip
\item Reality: poorest 20\% have 0\% wealth.
\item Model: poorest 20\% have 9\% wealth
\item Reality: richest 5\% have 50\% wealth.
\item Model: richest 5\% have 11\% wealth
\bigskip
\item How do we generate this?  
\begin{itemize}
\item Random discount factors
\item Differences in unemployed income
\end{itemize}
\item These can nail the distribution
\item With a more reasonable wealth distribution, nothing changes 
\end{itemize}
\end{frame}

\begin{frame}
\frametitle[alignment=center]{Aggregate Time Series}
\begin{itemize}
\bigskip
\item Lack of full insurance increases capital by 0.6\% in the baseline.
\bigskip
\item Up to 6.7\% with high risk aversion
\bigskip
\item Can get more hand-to-mouth with different $\beta$'s, aggregate no longer looks like PHI
\bigskip
\item Not many differences between representative agent and heterogeneous agent, except PIH-type behavior.
\end{itemize}
\end{frame}

\begin{frame}
\frametitle[alignment=center]{Conclusions}
\begin{itemize}
\bigskip
\item Novel way to introduce interacting agents.
\bigskip
\item Reminds us that bounded rationality w.r.t. expectations is very easy with Bellmans
\bigskip
\item No change from heterogeneous agents is a result!
\end{itemize}
\end{frame}

\begin{frame}
\frametitle[alignment=center]{New Ways}
\begin{itemize}
\bigskip
\item Auclert, Bard\'{o}czy, Rognlie, and Straub (2021).
\bigskip
\item Reminds us that bounded rationality w.r.t. expectations is very easy with Bellmans
\bigskip
\item No change from heterogeneous agents is a result!
\end{itemize}
\end{frame}



\begin{frame}
\frametitle[alignment=center]{Computational Course Takeaways}
\small
\begin{itemize}
\item Economic intuition frequently more helpful than computational skill in solving computational problems
\item Structural estimation on normal (well-behaved) problems is very easy (NFP)
\begin{enumerate}
\item Make guesses at parameters
\item Solve model
\item Calculate difference between moments and reality (or likelihood)
\item Go back to step 1 with this data, update parameters
\end{enumerate}
\item True in CGE models where you might (1) solve individual problems given w and $\theta$, (2) clear markets given L($\theta$), then calculate moments
\item True in dynamic choice models where you might (1) solve for Bellman given $\theta$  then choose $\theta$ to fit moments
\item True in heterogeneous-agent models where you (1) assume parameters, a law of motion (2) solve VFI given parameters, law of motion (3) simulate law of motion (4) return to step \#2 until law of motion is accurate description (5) return to step \#1 to change parameters.
\end{itemize}
\end{frame}




\end{document}